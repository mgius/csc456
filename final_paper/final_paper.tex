\documentclass[12pt]{article}

\usepackage{fullpage} % does reasonable things
\usepackage{setspace} % allows double spacing

\begin{document}
\title{``Frozen'' Linux}
\author{Mark Gius \\
        CPE456, Section 01}
\date{\today}
\maketitle

\begin{abstract}
When maintaining a fleet of public use personal computers, the systems
administrator is faced with the task of trying to protect the fleet from 
all manner of threats, ranging from the accidental to the malicious.  
The protective mechanism often takes the form of restricted or reduced 
access, preventing damage by forbidding access to dangerous functions.
This mechanism has a negative effect when a user attempts to use the system
in a manner that was not predicted by the systems administrator.
I propose a system that would allow the systems administrator to maintain
an acceptable level of control over the stability of the public fleet,
while still allowing users to configure and install the software that they
need.
\end{abstract}

\newpage
\doublespacing
\section{Introduction}
% TODO: Consistency among terms
The average computer user lacks the knowledge or will to safely maintain 
a general use personal computer.  The threats facing users today include 
self-inflicted damage such as accidental deletion and misconfiguration of the 
operating system or other programs, as well as the numerous types of malware
and other unwanted applications that are common today.

This problem is further multiplied in the case of public access computers, 
such as internet kiosks or laptop check-out stations.  In this case, not 
only are the users generally incapable of keeping themselves safe, they have
little to no motivation to protect assets which are not their responsibility.
In these scenarios, the public systems are often configured to disallow 
``unsafe'' actions, such as installing software, system configuration, and
installing hardware.

The locked down nature of these public machines has consequences, however.
Sometimes users intend to use these public or borrowed systems for purposes
that the administrator had not anticipated.  While most users use the public
systems for web browsing, an up-and-coming musician may want to save money by
checking out a laptop every week to record music.  The musician will need a
number of software and hardware settings that are not common.  Ordinarily, this
use case would be unsupported, due to being configured in a
``least-priviledge'' mode.  However, if it were possible to quickly and easily
revert changes made by the users, it would be reasonable to allow any user full
administrative priviledges over the checked out or public machine.  In this
way, the musician can use public resources exactly how he wishes, and the
technicians running the shared laptop program can trivially restore the laptops
to a known good state.

I set out to create a system whereby systems with administrative access enabled
can be provided to well-meaning users with advanced needs, without compromising
the long-term integrity of the shared systems. 

\section{Possible Implementations}
There are many possible avenues for implementing such a system.

One possibility is to distribute systems that boot from Linux LiveCDs, instead
of the local hard disk.  This option has a number of drawbacks which make it a
poor choice for the aspiring musician. For one, LiveCDs save all temporary
changes to an in-RAM filesystem, limiting the amount of usable space to do
work.  In the case of the musician, the music editing software (both on disk
and the running memory usage) and the recorded music files will all share
the same limited amount of space.  Without significant quantities of RAM, the
machines would not be usable by certain classes of users.

Another solution could keep a known good state on disk, with the ability to
intiate a ``re-image'' process to restore the system to a known-good state.
This would meet the requirements of our musician, but restoring the system
to a known-good state would be non-trivial and potentially time-consuming 
depending on the size of the disks.  Ideally, our system would support 
instant (or nearly so) reversion, in a manner that was trivial for a user
or administrator to initiate.

A better solution uses a modified form of the filesystem that most Linux 
LiveCDs present. Normally, a LiveCD merges its known-good CD filesystem 
with an in-RAM filesystem. To make this solution work, rather than 
merging the known good filesystem with an in-RAM filesystem, we would 
merge the known good filesystem with another on-disk filesystem.  We would
then have to ensure that the second on-disk filesystem is properly reset 
at an appropriate time, such as on every system boot.

\section{Implementation}

As we have seen, there are a number of avenues by which we can implement this 
system.  In accordance with the scope of this project, I elected to pursue
a userspace implementation, using standard linux filesystems and AUFS, a
union filesystem, to protect the system against all changes.

The basic idea is that a real root device and an empty filesystem would be
merged using AUFS, with the blank ``scratch'' disk mount on top of the real
root disk.  AUFS would consider the real root filesystem to be a read-only
branch of the merged filesystem, and all modifications to the merged filesystem
would be written to the top-level ``scratch'' disk. On boot, the data on the
scratch disk would be thrown away, and a new blank scratch disk would be merged
with the real root filesystem.

The behavior that the user would see is a fresh, consistent system on every
reboot, no matter what happens between boots.

\subsection{The ``Scratch'' Disk}
When choosing the filesystem for the ``scratch disk'', or the filesystem that
is mounted ``over'' the real root filesystem, I looked into two broad
categories.  The first were traditional linux filesystems, such as ext3 and
JFS, and the second were snapshotting filesystems, such as nilfs2.  

I investigated the nilfs filesystem as a possible avenue for the scratch disk,
hoping to use it's snapshotting and checkpoint features to allow an
administrator to easily modify the current state of the system without
significant effort.  With a snapshotting filesystem, an administrator could
make changes or updates to the system then create a snapshot.  The boot-time
reversion, instead of wiping away all the data on the filesystem, would simply
revert to the lastest snapshot.

Unfortunately, nilfs does not support this kind of reversion.  While nilfs does
support checkpoints and snapshots, as well as mounting those intermediate
points, it does not support read-write mounting of snapshots.  Even worse, when
snapshots and checkpoints are deleted, the current state of the filesystem is
unchanged, only the intermediate points are expunged from the filesystem. As a
result, nilfs offered no advantages over a traditional filesystem, and I
decided not to use it.

In the end, I chose JFS as my overlaid filesystem.  Based on my previous
experiences with Linux filesystems, I chose JFS because of its general speed,
especially in regard to large files, and its speed in formatting.  JFS
filesystems, even with extremely large disks of 1TB or more, format in a time
on the order of seconds.  Because the scratch disk will be reformatted after
every reboot, I felt that minimizing the amount of time necessary for each
reboot would be beneficial.

\subsection{Kernel and Initial Ram Disk}

Initially, I had hoped to use a stock Ubuntu kernel.  After some
experimentation I decided to compile my own kernel, even though the Ubuntu
kernel supports the AUFS filesystem.  The two compelling factors for compiling
my own kernel were filesystem support and userspace tools support.  The stock
Ubuntu kernel has support for JFS, but it is compiled as a module.  In order to
mount filesystems as a root filesystem, support for the filesystem must be
compiled directly into the kernel.

Ubuntu, like many Linux distributions, uses AUFS to build their Live CDs, but
don't include the userspace tools in their repositories.  Because the mainline
kernel (Linus Torvalds' tree) does not include any union filesystems, any
distribution using the AUFS code has applied patches directly to their source.
Ideally, I would match the AUFS userspace tools version to the AUFS version
applied to the kernel.  Rather than parse through the myriad of patches applied
to the Ubuntu kernel to find the correct version of AUFS to download, I opted
instead to download the latest version of the AUFS kernel and AUFS userspace
tools from the project webpage.\footnote{http://aufs.sourceforge.net/}.  

No modifications were made to the source of the AUFS enabled kernel, although
many features and drivers were disabled to speed up the compilation of the 
kernel.  

The initial ram disk (initrd) was modified slightly to change the behavior of mounting
the root filesystem.  The JFS filesystem format binary was added to the 
initrd to support reformatting the filesystem after every boot, and the 
script which is responsible for mounting the root filesystem was modified.

The original behavior of the root mounting script was to mount the root
filesystem in a particular location.  I added several lines which format the
scratch disk, mount the real root and scratch disk in temporary locations, then
union mount those two filesystems using AUFS in the location that the script
would normally place the real root filesystem.  The result of these
interactions is that the root filesystem that the initrd eventually hands off
to the system's init is the merged AUFS filesystem.

% TODO: more here?

\subsection{Known Vulnerabilities}
While this implementation is very effective against accidental damages to
the system, it does not hold up well against a concerted attack. There are two
primary methods by which an attacker could easily circumvent the protections
offered and create persistent changes to the system. For all of these attacks,
it is assumed that the attacker has obtained root privileges.

The first involves manipulating the disk block devices directly. 
Under normal circumstances, special files representing disks and partitions
are exposed to the user via the ``devfs'' special filesystem.  Because these 
block devices represent a lower level interface than that of the AUFS mounted
filesystem it is possible to read and write to the block devices directly, 
completely bypassing the protection offered by the AUFS union.

Using this vulnerability, an attacker could attempt to mount the real root
filesystem in a temporary folder and modify the contents there.  This attack 
is nontrivial, primarily because the Linux kernel will not normally mount
a filesystem twice.  Although there are mount options that allow you to 
``mount'' a filesystem in multiple locations (bind, remount), 
these options require access to theactual mount point of the filesystem to 
be multiply mounted. Because the true mount point of the root filesystem is
not visible after the initial ram disk calls init it becomes much harder to
exploit this vulnerability in this manner, although by no means impossible.

Another exploit of this vulnerability involves duplicating the filesystem on
the disk using the \emph{dd} command.  After mounting the duplicated system and
making modifications, the attacker could then write the modified filesystem
back to disk.  While I have been unable to exploit this vulnerability myself
(it often leaves the filesystem in a dirty or unmountable state), a dedicated
attacker could eventually develop a technique for reliably exploiting this
vulnerability.

The second vulnerability revolves around the boot loader.  By default, the 
GRUB boot loader does not have a password, and allows users to edit the boot
loader options.  Because the implementation relies upon a specific initrd
to set up the AUFS filesystem, an attacker could specify another initrd 
that doesn't protect the filesystem.  This attack could be mitigated by 
adding a password to the boot loader, and only leaving the AUFS mounting 
initrd's on the filesystem.

The third vulnerability is that there are no checks to verify that the root
filesystem is unchanged.  Using this knowledge, an attacker could boot the 
system using a Linux LiveCD and modify the root filesystem from there.  
This vulnerability could be mitigated by disallowing booting from media other 
than the local disk, and storing the protected in a monitored environment. 
The system could also be modified to verify that the last mount event 
was performed at the right time and by the right system, although such a check
would be easily fooled by a well crafted attack.  

\section{Future Improvements}
One significant improvement that could be made would be to use a snapshotting
filesystem that supports reversion for the scratch disk, instead of JFS.  None
of the snapshotting filesystems that I tested supported this functionality, but
the behavior could most likely be patched in. By doing this, the administrator
could create a new default snapshot that would be reverted to after each
reboot, greatly reducing the complexity of updating the base image.

Another option would be to implement the entire system as a kernel block
device.  Instead of using a union at the filesystem level, the union would
occur between two disks or partitions.  This would improve security
dramatically, as the unioned block device would be presented through
\emph{devfs} instead of the real block devices, and the merged block devices
could be removed.

\section{Conclusion}
The proposed implementation greatly improves the usability of public use
terminals, especially public access laptops such as those provided by many
public libraries.  The system has much room for improvement, especially in
regard to preventing malicious attacks and ease of upgrade.  

\nocite{*}
\newpage

\bibliographystyle{plain}
\bibliography{final_paper}
\end{document}
