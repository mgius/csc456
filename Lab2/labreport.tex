\documentclass[11pt]{article}

% Use wide margins, but not quite so wide as fullpage.sty
\marginparwidth 0.5in 
\oddsidemargin 0.25in 
\evensidemargin 0.25in 
\marginparsep 0.25in
\topmargin 0.25in 
\textwidth 6in \textheight 8 in
% That's about enough definitions

\usepackage{amsmath}
\usepackage{upgreek}

\begin{document}
\author{Mark Gius}
\title{Lab 2: Cryptography Exercises}
\maketitle

\begin{enumerate}

\item % Problem 1: Decode this string
\begin{tabular}{p{3cm} p{1cm} p{8cm}}
\begin{verbatim}
    S P I E A
    B C D F G
    H K L M N
    O Q R T U
    V W X Y Z
\end{verbatim} & &
\begin{verbatim}
OMEUP ADQFY VTOZS FIYAM NZIUL DUNEP MV
THATS ECRET YOUVE BXEEN GUARD INGIS NT
\end{verbatim}
The ``X'' in ``BEEN'' is a padded character to force the message to be an 
even number of characters
\end{tabular}

\item % Problem 2: # of keys
Assuming that each person wants to communicate securely with everybody else,
and each person is willing to send a separately encrypted message if a message
needs to be sent to more than one person, then there will need to be one key
for each unique pairing.  This works out to 
\[\frac{(m-1) m}{2}\]
which also happens to be the number of edges in a fully connected graph 
with $m$ nodes.

If sending multiple copies of a message is disallowed, this increases the
number of keys required.  For every possible group of persons greater than 2,
a key is required.  Looking up the equation for $m$ choose $k$ gives me

\[\frac{m!}{k!(m-k)!}\]

So now I'll just sum all of the possible groupings for a group of size m

\[
\sum_{k=2}^{m} \frac{m!}{k!(m-k)!}
\]

Running the numbers for a group of size 4, I get $6 + 4 + 1 = 11$ keys to 
allow every combination of persons to securely communicate amongst each other.

\item % what if #2 was PKI

If the group of people in problem 2 were allowed to use a public key 
cryptosystem, the number of keys required would be $m * 2$. Each person would
need their own Public key and their own Private key.

\item % untrustworthy investors and stockbrokers

With public key cryptography (and sufficiently strong keys), we can be 
absolutely certain which person sent the messages.  Let us assume that all
keys are sufficiently long that they cannot be trivially cracked, and that
none of the private keys have been compromised. 

Let us assume that the stockbrocker, Sally, is trying to boost his numbers by
ordering stocks in the name of the investor, Ian.  If the investment firm's
policy states that purchases can only be made by email signed by an investor,
then we can easily determine whether or not Ian sent the order.  Because a
message signed with one person's private key can only be decrypted by their
public key, any orders that the stockbroker acts on must be decrypted using
the public key.  Orders that are signed by any key other than Ian's private
key will not decrypt using his public key, and Ian can plausibly deny having
sent that order. 

In the case where the investor has lost money and is trying to deny having
sent the buy order, we can use the same policy as above to provide proof of an
order having been requested.  Since only Ian can sign messages using his
private key, any orders that can be succesfully decrypted using Ian's public
key can be proven to have originated from him.

So long as policy requires that all investor orders must be signed by the 
investor's private key, the both parties in the exchange can be protected from
fraudulent claims about the orders.

\item % m = 10, e = 7, n = 55
\[
m = 10 \quad e = 7 \quad n = 55
\]
\begin{align*}
m^e \bmod n & = m' \\
10^7 \bmod 55 & = 10 \\
m' & = 10
\end{align*}

\item % break above
\[
m = 10 \quad e = 7 \quad n = 55 \quad C = 35
\]
\begin{align*}
n & = pq \\
55 & = pq \\
55 & = 5 * 11 \\
p & = 5 \\
q & = 11 \\
\end{align*}
\begin{align*}
\upvarphi(n) & = (p - 1)(q - 1) \\
\upvarphi(n) & = 4 * 10 \\
\upvarphi(n) & = 40
\end{align*}
\begin{align*}
ed & \equiv \pmod{40} \\
7d & \equiv \pmod{40} \\
7 * 23 & \equiv \pmod{40} \\
d & = 23
\end{align*}
\begin{align*}
C & = 35 \\
C^d \bmod n & = M \\
35^{23} \bmod 55 & = M \\
30 & = M
\end{align*}
\end{enumerate}

\end{document}
