\documentclass[11pt]{article}

\usepackage{fullpage} % does reasonable things
% Use wide margins, but not quite so wide as fullpage.sty
%\marginparwidth 0.5in 
%\oddsidemargin 0.25in 
%\evensidemargin 0.25in 
%\marginparsep 0.25in
%\topmargin 0.25in 
%\textwidth 6.5in \textheight 9.0in
% That's about enough definitions

%\usepackage{multirow}
%\usepackage{tabularx}
%\usepackage{longtable}

%\begin{document}
%\author{Mark Gius}
%\title{Project Proposal: ``Frozen'' Linux Deployments}
%\maketitle
\begin{document}
\hfill\vbox{\hbox{Gius, Mark}
            \hbox{Cpe 456, Section 01} 
            \hbox{Project Proposal}   
            \hbox{\today}}\par

\bigskip
\centerline{\Large\bf Project Proposal: ``Frozen'' Linux Deployments}\par
\bigskip

\section{Introduction}
I propose to implement a system that prevents any changes to disk from
persisting across system reboot. Such a system would be extremely resistant to
damage by incompetence, and moderately resistent to damage by malice.  In this
way, we can create a system that can be safely given to untrusted individuals
without compromising the integrity of the system.

\section{Implementation and Scope}
I intend to use a combination of tools to modify a linux system to prevent
local changes from persisting after a reboot, without reinstalling or
reimaging the system.

This implementation is not meant to be bulletproof.  Because my implementation
will be primarily be in userspace, a user with superuser priviledges will
still be able to manipulate local resources via the block devices in /dev. A
more robust implementation could be implemented as a kernel module that
catches calls to block devices and redirects them to other block devices rather 
than overlaying a temporary filesystem over the root filesystem.

Ideally, this implementation could be used to ``secure'' systems that are to
be given to users who require moderate amounts of administrative power.
Examples include access to install packages or manipulate audio devices
directly.  As long as these users do not gain access to the block devices for
the local disks any damage they may cause would be transient. 

\section{Similar Implementations}
I am aware of two similar implementations of ``frozen'' operating system 
deployments.  

The first is a proprietary product sold by 
Faronics\footnote{http://www.faronics.com} called ``Deep Freeze.'' The software is
available for Windows, OSX, and Linux, although I have only seen it in use on
the Windows platform. 

The second implementation can be found on almost every Linux ``LiveCD.'' When
the operating system on the CD is loaded, the filesystem on the CD is merged
with a read/write filesystem that is resident in RAM, allowing the user to
``write'' files and change settings on the system.  These changes do not
persist, and the system is immutable because the media that the system files
are loaded from is read-only.  

\end{document}
